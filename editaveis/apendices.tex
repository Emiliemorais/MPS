\begin{apendices}
 

 \chapter{Plano de Trabalho - Projeto de Melhoria de Processos - Equipe SiGA}

\section{Escopo}

\subsection{Objetivos}

\textbf{Necessidades do negócio:} 
Melhorar a qualidade dos produtos entregues.

\textbf{Motivações para melhoria:} 
\begin{itemize}
	\item Baixa qualidade dos produtos entregues;
	\item Dificuldades na implantação;
	\item Despadronização das formas de trabalho;
	\item Falta de gerenciamento dos riscos.
\end{itemize}

\textbf{Objetivos de melhoria:} 
\begin{itemize}
	\item Formalizar o processo de desenvolvimento de software;
	\item Padronizar a execução das atividades do processo;
	\item Otimizar a atividade de implantação;
	\item Estabelecer métricas para o processo;
	\item Gerenciar os riscos dos projetos.
\end{itemize}

\subsection{Produtos Relevantes}

	Os produtos relevantes para o projeto de melhoria de processos da equipe SiGA são:
	
	\begin{itemize}
		\item Plano de Projeto de MPS;
		\item O processo de desenvolvimento de software;
		\item Cronograma;
		\item Estrutura analítica de risco;
		\item Relatório de Acompanhamento do projeto de MPS;
		\item Lista de melhorias.
	\end{itemize}


\section{Recursos}

\subsection{Recursos Humanos}

	Os recursos humanos alocados para o projeto são os dois desenvolvedores com dedicação total. Para orientação do projeto será alocado um orientador com disponibilidade para o projeto de 2 vezes por semana.


\subsection{Recursos Materiais e de Infraestrutura}

	A equipe já possui os recursos materiais e de infraestrutura necessários para execução do projeto.

\section{Cronograma}

\section{Riscos}

\section{Plano de Comunicações Relevantes}

	\begin{itemize}
		\item Aprovação do processo definido;
		\item Acompanhamento do projeto;
		\item Divulgação dos resultados;

	\end{itemize}

\section{Plano de Monitoramento}
	
	O monitoramento do projeto de MPS será realizado semanalmente juntamente com o orientador.


 \chapter{Processo de Desenvolvimento Atual}
\label{ap:processo_atual}

Neste capítulo estão definidos os papéis, atividades e ferramentas utilizadas no processo de desenvolvimento atual do SiGA.

\section*{Papéis}

O processo é realizado pelos seguintes papéis:

\begin{itemize}
 \item \textbf{Gerente/Analista de Requisitos:} Responsável por levantar os requisitos com o cliente, gerenciar a equipe, gerenciar o custo do projeto, 
 gerenciar o cronograma do projeto e realizar a implantação do sistema;
 \item \textbf{Desenvolvedores:} Desenvolver e testar o sistema;
 \item \textbf{Time:} Analisar os requisitos;
\end{itemize}

\section*{Ferramentas}

O processo é apoiado pelas seguintes ferramentas:

\begin{itemize}
 \item \textbf{Sublime Text:} Editor de texto utilizado para escrita do código; 
 \item \textbf{phpMyAdmin:} Gerenciador de banco de dados;
 \item \textbf{Github}: Utilizado para versionamento de código e gerenciamento de requisitos;
 \item \textbf{Waffle}: Utilizado para gerenciamento de requisitos de acordo com Kanban \cite{kanban};
\end{itemize}

\section*{Atividades}

O processo é composto de dez atividades que estão detalhadas nas tabelas abaixo.

\begin{table}[]
\centering
\caption{Descrição da atividade - Levantar Requisitos}
\begin{tabular}{|l|l|}
\hline
\textbf{Atividade}     & Levantar Requisitos                                                                                                                                                                                    \\ \hline
\textbf{Objetivo}      & Conhecer as necessidades do cliente a serem contempladas pelo sistema                                                                                                                                  \\ \hline
\textbf{Entradas}      & Escopo do projeto                                                                                                                                                                                      \\ \hline
\textbf{Saídas}        & Requisitos                                                                                                                                                                                             \\ \hline
\textbf{Responsável}   & Gerente/Analista de Requisitos                                                                                                                                                                         \\ \hline
\textbf{Procedimentos} & \begin{tabular}[c]{@{}l@{}}Não se aplica.\\ \\ Observações: A execução desta atividade não demanda procedimentos \\ definidos pois os requisitos podem ser levantados de diversas formas.\end{tabular} \\ \hline
\end{tabular}
\end{table}

\begin{table}[]
\centering
\caption{Descrição da atividade - Validar Requisitos}
\begin{tabular}{|l|l|}
\hline
\textbf{Atividade}     & Validar Requisitos                                                                                                                                                                                    \\ \hline
\textbf{Objetivo}      & Validar os requisitos levantados com o cliente                                                                                                                                                        \\ \hline
\textbf{Entradas}      & Requisitos                                                                                                                                                                                            \\ \hline
\textbf{Saídas}        & Requisitos refinados                                                                                                                                                                                  \\ \hline
\textbf{Responsável}   & Gerente/Analista de Requisitos                                                                                                                                                                        \\ \hline
\textbf{Procedimentos} & \begin{tabular}[c]{@{}l@{}}Não se aplica.\\ \\ Observações: A execução desta atividade não demanda procedimentos \\ definidos pois os requisitos podem ser validados de diversas formas.\end{tabular} \\ \hline
\end{tabular}
\end{table}

\begin{table}[]
\centering
\caption{Descrição da atividade - Priorizar Requisitos}
\begin{tabular}{|l|l|}
\hline
\textbf{Atividade}     & Priorizar Requisitos                                     \\ \hline
\textbf{Objetivo}      & Priorizar os requisitos definidos para o desenvolvimento \\ \hline
\textbf{Entradas}      & Requisitos                                               \\ \hline
\textbf{Saídas}        & Requisitos priorizados (a serem desenvolvidos na Sprint) \\ \hline
\textbf{Responsável}   & Gerente/Analista de Requisitos                           \\ \hline
\textbf{Procedimentos} & Não se aplica.                                           \\ \hline
\end{tabular}
\end{table}

\begin{table}[]
\centering
\caption{Descrição da atividade - Especificar e Documentar Requisitos}
\begin{tabular}{|l|l|}
\hline
\textbf{Atividade}     & Especificar e Documentar Requisitos                                                                                                                                                                                       \\ \hline
\textbf{Objetivo}      & \begin{tabular}[c]{@{}l@{}}Especificar os requisitos levantados detalhando a funcionalidade\\ e os critérios de aceitação e estimar.\end{tabular}                                                                         \\ \hline
\textbf{Entradas}      & Requisitos                                                                                                                                                                                                                \\ \hline
\textbf{Saídas}        & Issues                                                                                                                                                                                                                    \\ \hline
\textbf{Responsável}   & Time                                                                                                                                                                                                                      \\ \hline
\textbf{Procedimentos} & \begin{tabular}[c]{@{}l@{}}1. Discutir os requisitos levantados\\ 2. Escrever a história de usuário\\ 3. Escrever os critérios de aceitação\\ 4. Definir o responsável\\ 4. Definir a pontuação para a issue\end{tabular} \\ \hline
\end{tabular}
\end{table}
 
\chapter{Avaliação do Processo atual} \label{ap:avaliacao}

A avaliação foi realizada com base nas práticas, resultados e processos escolhidos do XP, Scrum, 12207 e CMMI. 
Para cada um desses elementos foi avaliada se é implementado (I), parcialmente implementado (PI) ou não implementado (NI) no processo
atual. 

\begin{table}[H]
\centering
\caption{Avaliação do processo de acordo com os processos escolhidos do CMMI}
\resizebox{\textwidth}{!}{%
\begin{tabular}{|l|l|l|l|l|}
\hline
\multicolumn{1}{|c|}{\textbf{Área de Processo}} & \multicolumn{1}{c|}{\textbf{Meta Específica}}                                                                     & \multicolumn{1}{c|}{\textbf{Prática específica}}                                                                                & \multicolumn{1}{c|}{\textbf{Status}} & \multicolumn{1}{c|}{\textbf{Descrição da evidência}}                                                                                                                         \\ \hline
\multirow{7}{*}{Gestão de Configuração (CM)}    & \multirow{3}{*}{SG 1 Estabelecer Baselines}                                                                       & SP 1.1 Identificar Itens de Configuração                                                                                        & NI                                   & -                                                                                                                                                                            \\ \cline{3-5} 
                                                &                                                                                                                   & \begin{tabular}[c]{@{}l@{}}SP 1.2 Estabelecer um Sistema de \\ Gestão de Configuração\end{tabular}                              & I                                    & \begin{tabular}[c]{@{}l@{}}É usado o Github como sistema de \\ gestão configuração\end{tabular}                                                                              \\ \cline{3-5} 
                                                &                                                                                                                   & SP 1.3 Criar ou Liberar Baselines                                                                                               & PI                                   & São criadas tags no Github para Release                                                                                                                                      \\ \cline{2-5} 
                                                & \multirow{2}{*}{\begin{tabular}[c]{@{}l@{}}SG 2 Acompanhar e Controlar \\ Mudanças\end{tabular}}                  & \begin{tabular}[c]{@{}l@{}}SP 2.1 Acompanhar Solicitações \\ de Mudança\end{tabular}                                            & I                                    & Issues do Github                                                                                                                                                             \\ \cline{3-5} 
                                                &                                                                                                                   & SP 2.2 Controlar Itens de Configuração                                                                                          & I                                    & Github                                                                                                                                                                       \\ \cline{2-5} 
                                                & \multirow{2}{*}{SG 3 Estabelecer Integridade}                                                                     & \begin{tabular}[c]{@{}l@{}}SP 3.1 Estabelecer Registros de \\ Gestão de Configuração\end{tabular}                               & PI                                   & Apenas os elementos providos pelo Github                                                                                                                                     \\ \cline{3-5} 
                                                &                                                                                                                   & \begin{tabular}[c]{@{}l@{}}SP 3.2 Executar Auditorias de \\ Configuração\end{tabular}                                           & NI                                   & -                                                                                                                                                                            \\ \hline
\multirow{9}{*}{Integração de Produto (PI)}     & \multirow{3}{*}{\begin{tabular}[c]{@{}l@{}}SG 1 Preparar-se para Integração \\ de Produto\end{tabular}}           & \begin{tabular}[c]{@{}l@{}}SP 1.1 Determinar Sequência \\ de Integração\end{tabular}                                            & I                                    & \begin{tabular}[c]{@{}l@{}}A integração é sempre do incremento de \\ software mais novo com o antigo\end{tabular}                                                            \\ \cline{3-5} 
                                                &                                                                                                                   & \begin{tabular}[c]{@{}l@{}}SP 1.2 Estabelecer Ambiente de \\ Integração do Produto\end{tabular}                                 & PI                                   & \begin{tabular}[c]{@{}l@{}}Há um ambiente de integração utilizado \\ para o gerente, mas é inadequado\end{tabular}                                                           \\ \cline{3-5} 
                                                &                                                                                                                   & \begin{tabular}[c]{@{}l@{}}SP 1.3 Estabelecer Procedimentos e \\ Critérios para Integração do Produto\end{tabular}              & NI                                   & -                                                                                                                                                                            \\ \cline{2-5} 
                                                & \multirow{2}{*}{\begin{tabular}[c]{@{}l@{}}SG 2 Assegurar Compatibilidade \\ das Interfaces\end{tabular}}         & \begin{tabular}[c]{@{}l@{}}SP 2.1 Revisar Descrições de Interfaces \\ para Assegurar Completude\end{tabular}                    & PI                                   & \begin{tabular}[c]{@{}l@{}}As interfaces necessárias são descritas nas issues, \\ mas não há um procedimento explícito de revisão\end{tabular}                               \\ \cline{3-5} 
                                                &                                                                                                                   & SP 2.2 Gerenciar Interfaces                                                                                                     & PI                                   & Há uma gerência mínima através do Composer                                                                                                                                   \\ \cline{2-5} 
                                                & \multirow{4}{*}{\begin{tabular}[c]{@{}l@{}}SG 3 Montar Componentes do \\ Produto e Entregar Produto\end{tabular}} & \begin{tabular}[c]{@{}l@{}}SP 3.1 Confirmar se os Componentes \\ do Produto estão Prontos para \\ serem Integrados\end{tabular} & PI                                   & \begin{tabular}[c]{@{}l@{}}A confirmação é realizada através da \\ disponibilização do código na master, \\ ou seja, está sob subjetividade dos desenvolvedores\end{tabular} \\ \cline{3-5} 
                                                &                                                                                                                   & SP 3.2 Montar Componentes do Produto                                                                                            & I                                    & \begin{tabular}[c]{@{}l@{}}O incremento é adicionado ao sistema\\  em ambiente de produção\end{tabular}                                                                      \\ \cline{3-5} 
                                                &                                                                                                                   & \begin{tabular}[c]{@{}l@{}}SP 3.3 Avaliar Componentes de \\ Produto Montados\end{tabular}                                       & PI                                   & \begin{tabular}[c]{@{}l@{}}A avaliação não é formal, por isso são \\ encontrados vários problemas\end{tabular}                                                               \\ \cline{3-5} 
                                                &                                                                                                                   & \begin{tabular}[c]{@{}l@{}}SP 3.4 Empacotar e Entregar Produto \\ ou Componente de Produto\end{tabular}                         & NI                                   & -                                                                                                                                                                            \\ \hline
\end{tabular}%
}
\end{table}
\begin{table}[]
\centering
\caption{Avaliação do processo de acordo com os processos escolhidos da 12207}
\resizebox{\textwidth}{!}{%
\begin{tabular}{|l|l|l|l|}
\hline
\multicolumn{1}{|c|}{\textbf{Processo}} & \multicolumn{1}{c|}{\textbf{Resultados}} & \multicolumn{1}{c|}{\textbf{Status}} & \multicolumn{1}{c|}{\textbf{Descrição da evidência}} \\ \hline
\multirow{4}{*}{\begin{tabular}[c]{@{}l@{}}Garantia de Qualidade \\ de Software\end{tabular}} & \begin{tabular}[c]{@{}l@{}}Uma estratégia para conduzir a garantia \\ de qualidade é desenvolvida\end{tabular} & NI & - \\ \cline{2-4} 
 & \begin{tabular}[c]{@{}l@{}}Evidência da qualidade do software \\ é produzida e mantida\end{tabular} & NI & - \\ \cline{2-4} 
 & \begin{tabular}[c]{@{}l@{}}Problemas e não-conformidades com os \\ requisitos são identificados e documentados\end{tabular} & PI & \begin{tabular}[c]{@{}l@{}}Os problemas e não-conformidades \\ são identificados, mas raramente documentados\end{tabular} \\ \cline{2-4} 
 & \begin{tabular}[c]{@{}l@{}}A aderência dos produtos, processos e \\ atividades aos padrões aplicáveis, \\ procedimentos e requisitos é verificada\end{tabular} & NI & - \\ \hline
\multirow{6}{*}{Gestão de Configuração} & \begin{tabular}[c]{@{}l@{}}Uma estratégia de gestão de configuração \\ é definida\end{tabular} & NI & - \\ \cline{2-4} 
 & Itens de configuração são definidos & NI & - \\ \cline{2-4} 
 & Baselines são estabelecidas & NI & - \\ \cline{2-4} 
 & \begin{tabular}[c]{@{}l@{}}Mudanças nos itens de configuração \\ são gerenciadas e controladas\end{tabular} & NI & - \\ \cline{2-4} 
 & \begin{tabular}[c]{@{}l@{}}A configuração de itens entregues \\ é controlada\end{tabular} & NI & - \\ \cline{2-4} 
 & \begin{tabular}[c]{@{}l@{}}O estado dos itens de configuração é \\ disponibilizado ao longo do ciclo de vida.\end{tabular} & NI & - \\ \hline
\multirow{7}{*}{Gestão de Medição} & As necessidades de informação são identificadas & NI & - \\ \cline{2-4} 
 & \begin{tabular}[c]{@{}l@{}}Um conjunto apropriado de métricas, \\ dirigido pelas  necessidades de informação \\ são identificados  e/ou desenvolvidos\end{tabular} & NI & - \\ \cline{2-4} 
 & \begin{tabular}[c]{@{}l@{}}Atividades de medição são identificas \\ e planejadas\end{tabular} & NI & - \\ \cline{2-4} 
 & \begin{tabular}[c]{@{}l@{}}Os dados necessários são coletados, \\ armazenados, analisados e os\\ resultados interpretados\end{tabular} & NI & - \\ \cline{2-4} 
 & \begin{tabular}[c]{@{}l@{}}Produtos de informação são utilizados \\ para suportar decisões e prover uma \\ base objetiva para a comunicação\end{tabular} & NI & - \\ \cline{2-4} 
 & \begin{tabular}[c]{@{}l@{}}O Processo de Medição e as métricas \\ são avaliados\end{tabular} & NI & - \\ \cline{2-4} 
 & \begin{tabular}[c]{@{}l@{}}Melhorias são comunicadas ao \\ responsável pelo Processo de Medição.\end{tabular} & NI & - \\ \hline
\multirow{4}{*}{\begin{tabular}[c]{@{}l@{}}Teste de Qualificação \\ de Software\end{tabular}} & \begin{tabular}[c]{@{}l@{}}Critérios de aceitação para o produto de \\ software são desenvolvidos para \\ demonstrar a conformidade com os requisitos\end{tabular} & I & \begin{tabular}[c]{@{}l@{}}Os critérios de aceitação são \\ escritos nas issues do Github\end{tabular} \\ \cline{2-4} 
 & \begin{tabular}[c]{@{}l@{}}O software integrado é verificado \\ utilizando os  critérios definidos\end{tabular} & NI & - \\ \cline{2-4} 
 & Os resultados dos testes são armazenados & NI & - \\ \cline{2-4} 
 & \begin{tabular}[c]{@{}l@{}}Uma estratégia de regressão é \\ desenvolvida para retestar o software\\  quando há mudanças.\end{tabular} & PI & \begin{tabular}[c]{@{}l@{}}Não há uma boa estratégia de regressão definida, \\ todavia com mudanças grandes há testes\\  funcionais não automatizados\end{tabular} \\ \hline
\end{tabular}%
}
\end{table}

\begin{table}[]
\caption{Avaliação do processo de acordo com as práticas ágeis escolhidas}
\resizebox{\textwidth}{!}{%
\begin{tabular}{|l|l|l|}
\hline
\multicolumn{1}{|c|}{\textbf{Prática}} & \multicolumn{1}{c|}{\textbf{Status}} & \multicolumn{1}{c|}{\textbf{Descrição da evidência}} \\ \hline
Rápido Feedback & I & A comunicação entre o time e o cliente é constante \\ \hline
Manter a simplicidade & I & Há esforço da equipe para manter a simplicidade \\ \hline
Mudanças incrementais & I & São gerados incrementos a cada semana \\ \hline
Trabalho de qualidade & PI & \begin{tabular}[c]{@{}l@{}}Há esforço da equipe para manter a qualidade, todavia não há uma boa \\ suíte de testes, e há problemas na integração\end{tabular} \\ \hline
Testes & PI & Há uma suíte de testes, porém com poucos testes \\ \hline
Refatoração & PI & Não é uma prática constante, mas refatorações são realizadas \\ \hline
Integração contínua & NI & - \\ \hline
Releases curtas & I & \begin{tabular}[c]{@{}l@{}}Não há uma divisão exata em Releases, todavia há entrega \\ contínua de incrementos de software\end{tabular} \\ \hline
\end{tabular}%
}
\end{table}


\end{apendices}
