\chapter{Introdução}

O Instituto de Letras (IL) da Universidade de Brasília \footnote{http://www.il.unb.br/} além de ser o departamento responsável por oferecer as disciplinas do
curso de Letras, também oferece cursos de pós-graduação: mestrado e doutorado em Linguística; mestrado e doutorado em Literatura, 
mestrado em Linguística Aplicada e em Estudos da Tradução. 

Com o intuito de gerenciar seus assuntos acadêmicos e financeiros, principalmente para os cursos de pós-graduação, o IL 
possui a necessidade de um sistema de software capaz de apoiar essa gestão. Nesse contexto, no ano de 2014 foi iniciado o 
desenvolvimento do Sistema Integrado de Gestão Acadêmica (SiGA).

Atualmente a equipe de desenvolvimento deste sistema é composta por dois desenvolvedores e um gerente/analista
de requisitos e o sistema encontra-se em produção. Dessa forma, o sistema está sob manutenção e evolução.

O processo de desenvolvimento (evolução) do sistema cujo objetivo é definir atividades de requisitos, construção, teste e implantação 
é baseado nas metodologias ágeis, porém ainda não está bem consolidado. Com isso em mente, o presente trabalho visa elaborar uma melhoria para o processo de desenvolvimento
do SiGA com o intuito de formalizá-lo e resolver alguns problemas encontrados. Embora o sistema esteja sob manutenção, o processo
de recebimento, priorização e resolução das demandas de manutenção não será escopo deste trabalho, considerando os recursos previstos.

Este trabalho está organizado em quatro capítulos. No capítulo \ref{cap:abordagem} é apresentada a abordagem escolhida
para execução do trabalho. Nos capítulos \ref{cap:iniciacao}, \ref{cap:diagnostico} e \ref{cap:planejamento}
estão documentados os resultados da fase de Iniciação, Diagnóstico e Planejamento, respectivamente.
