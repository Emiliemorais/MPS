\chapter{Iniciação} \label{cap:iniciacao}

A melhoria da qualidade dos produtos entregues tem sido uns dos principais focos da equipe, pois 
foi percebido que um dos problemas enfrentados 
no desenvolvimento do SiGA atualmente é a garantia da qualidade
do produto entregue continuamente. O tamanho da equipe, dedicação parcial no
desenvolvimento e tempo para entregas são fatores que tem influenciado em uma baixa garantia da qualidade. 
Além disso a falta de monitoramento e acompanhamento do projeto também impactam no controle da qualidade. 

Considerando estes problemas, nesta fase foram identificadas as motivações de melhoria bem como os objetivos relacionados. 
A infraestrutura e recursos necessários também foram estabelecidos. 
Os elementos definidos nesta fase constam no Plano de Projeto que encontra-se no Apêndice \ref{plano_de_projeto}. 

Além disso, foram estabelecidos os princípios-guia deste projeto de melhoria, são eles:

\begin{itemize}
 \item Manter a agilidade do processo;
 \item Manter o foco na qualidade do produto entregue;
\end{itemize}

Com a realização da fase de Iniciação, foi percebido que há recursos disponíveis e um clima favorável para a realização do projeto de melhoria,
além disso a proposta do projeto obteve aprovação por parte do Gerente. 