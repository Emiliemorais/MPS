\chapter{Abordagens de melhoria} \label{cap:abordagem}

Existem diversas abordagens e modelos existentes para a realização de uma melhoria de processo. Porém todas elas possuem fases em comum 
que a caracterizam como um modelo de melhoria, os modelos QIP e IDEAL são bons exemplos disso. O início de um projeto de melhoria de processo é caracterizado pela análise de viabilidade
considerando os recursos, infraestrutura, clima e objetivos e necessidades estabelecidos. Também é importante a caracterização do processo
atual e do estado desejável para aquele processo, isso é mapeado por uma fase de Diagnóstico. 

Além disso, é essencial o planejamento do projeto de melhoria, assim como em qualquer projeto. Nesse momento são estabelecidos as ações 
necessárias para se alcançar as melhorias pretendidas. Na execução dessas ações é de suma importância o monitoramento, pois é com ele 
que é possível identificar a eficácia das melhorias sugeridas e os pontos passíveis de mudanças. 

\section{Melhorias de processos de software}

Tratando-se de melhorias de processos de software (MPS) existem processos que auxiliam e dão diretrizes para que o desenvolvimento
de software possua uma boa qualidade. Os modelos de maturidade são bons exemplos disso. 

De acordo com \citeonline{pressman}, um modelo de maturidade consiste em uma série de indicadores 
de qualidades de processo , fornecendo tal indicação em nível de maturidade. Ele afirma que “sua finalidade é proporcionar 
uma indicação geral da “maturidade do processo” exibida por uma organização de software”. Um modelo em geral estabelece
processos padronizados que uma organização deve seguir. A adoção destes processos em um organização demonstram a maturidade daquele processo.
Embora o objetivo final seja o estabelecimento dessa maturidade, a adoção destes processos acarreta na melhoria da qualidade
dos processos da organização.

Para os processos de software existem dois modelos CMMI e o MPS.BR. “O CMMI ® (Capability Maturity Model ® Integration – Modelo 
Integrado de Maturidade e de Capacidade) é um modelo de maturidade para melhoria de processo, destinado ao desenvolvimento de produtos 
e serviços, e composto pelas melhores práticas associadas a atividades de desenvolvimento e de manutenção que cobrem o ciclo de vida 
do produto desde a concepção até a entrega e manutenção.” \cite{cmmi}

O MPS.BR é um programa que visa a Melhoria de Processos de Software e Serviços em duas metas: meta técnica e meta de negócio. 
A meta técnica está relacionada com a melhoria do modelo MPS e meta de negócio consiste na difusão do modelo para auxiliar pequenas, 
médias (foco) e grandes organizações. \cite{mps_br}

Os dois modelos fornecem guias para que as organizações possam implementar os processos estabelecidos, dessa forma, os modelos
não são indicados apenas para as organizações que desejam ser avaliadas em nível de maturidade, mas também para melhoria dos processos
de forma a garantir e controlar a qualidade dos processos e produtos. Além dos modelos de maturidades, 
há a ISO 12207/2008 \cite{12207} que estabelece processos relacionados ao ciclo de vida de sistemas e software, desse modo, podendo
ser utilizada para orientação de equipes e organizações dos processos que devem ser implantados. 

\section{Abordagens adotadas}

Considerando o contexto em que este projeto de melhoria está inserido foram avaliadas as abordagens e modelos que seriam adotados. Para
a organização do projeto de melhoria foi escolhido o modelo IDEAL \cite{ideal}, por ser um modelo bastante difundido, simples e que 
cumpriria com as necessidades de organização das fases e atividades do processo de melhoria. É um modelo que está estruturado em cinco fases: 
Iniciação, Diagnóstico, Planejamento, Ação e Aprendizado e para cada fase possui atividades e produtos associados. 

Para as melhorias no processo de desenvolvimento do SiGA optou-se pela utilização da ISO 12207/2008, identificando os processos necessários
de acordo com os objetivos de melhoria identificados. Além disso,
considerando a metodologia de desenvolvimento já utilizada pela equipe foi definido que as práticas ágeis provenientes de diversos
métodos também seriam adotadas. Os processos da ISO 12207/2008 e as práticas ágeis adotadas estão descritas no Capítulo \ref{cap:planejamento}. 





