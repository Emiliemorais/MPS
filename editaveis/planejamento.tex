\chapter{Planejamento} \label{cap:planejamento}

Afim de consolidar o planejamento do projeto de MPS deste trabalho foi estabelecido um Plano de Projeto que pode ser visto no
Apêndice \ref{plano_de_projeto}. Neste plano consta o planejamento de recursos humanos e o planejamento do tempo materializado
pelo cronograma. Para este projeto não foram considerados os custos envolvidos. 


\section{Embasamento da abordagem}

Para guiar a definição das ações de melhoria, alinhado com os objetivos de melhoria,
foram considerados os processos de Garantia de Qualidade de Software, de Gestão de Configuração,
de Gestão de Medição e Teste de Qualificação de Software da norma ISO/IEC 12207/2008 \cite{12207}, porque o propósito de ambos está 
de acordo com os objetivos de melhoria do projeto, podendo contribuir com atividades para o processo de melhoria.

De acordo com a norma ISO/IEC 12207/2008 \cite{12207}, os processos podem ser resumidos em seus respectivos propósitos e resultados:\\

\noindent
\textbf{Processo de Garantia de Qualidade de Software}
\begin{itemize}
    \item \textbf{Propósito}: \emph{Assegurar que os produtos de trabalho e processos estejam de acordo com os planos e provisões predefinidos.}
    \item \textbf{Resultados}:
        \subitem - Uma estratégia para conduzir a garantia de qualidade é desenvolvida;
	\subitem - Evidência da qualidade do software é produzida e mantida;
	\subitem - Problemas e não-conformidades com os requisitos são identificados e documentados;
	\subitem - A aderência dos produtos, processos e atividades aos padrões aplicáveis, procedimentos e requisitos é verificada.
\end{itemize}


\noindent
\textbf{Processo de Gestão de Configuração}
\begin{itemize}
    \item \textbf{Propósito}: \emph{Estabelecer e manter a integridade de todos os produtos identificados do projeto ou processo
	   e disponibilizá-los às partes interessadas.}
    \item \textbf{Resultados}:
        \subitem - Uma estratégia de gestão de configuração é definida;
	\subitem - Itens de configuração são definidos;
	\subitem - Baselines são estabelecidas;
	\subitem - Mudanças nos itens de configuração são gerenciadas e controladas;
	\subitem - A configuração de itens entregues é controlada;
	\subitem - O estado dos itens de configuração é disponibilizado ao longo do ciclo de vida.
\end{itemize}


\noindent
\textbf{Processo de Gestão de Medição}
\begin{itemize}
    \item \textbf{Propósito}: \emph{Coletar, analisar e comunicar dados relacionados aos produtos desenvolvidos e 
	  os processos implementados na organização, para dar suporte ao gerenciamento efetivo e demonstrar a qualidade dos produtos.}
    \item \textbf{Resultados}:
        \subitem - As necessidades de informação são identificadas;
	\subitem - Um conjunto apropriado de métricas, dirigido pelas necessidades de informação são identificados e/ou desenvolvidos;
	\subitem - Atividades de medição são identificas e planejadas;
	\subitem - Os dados necessários são coletados, armazenados, analisados e os resultados interpretados;
	\subitem - Produtos de informação são utilizados para suportar decisões e prover uma base objetiva para a comunicação;
	\subitem - O Processo de Medição e as métricas são avaliados;
	\subitem - Melhorias são comunicadas ao responsável pelo Processo de Medição.
\end{itemize}

\noindent
\textbf{Processo de Teste de Qualificação de Software}
\begin{itemize}
    \item \textbf{Propósito}: \emph{Confirmar que o produto integrado de software está de acordo com os requisitos definidos.}
    \item \textbf{Resultados}:
        \subitem - Critérios de aceitação para o produto de software são desenvolvidos para demonstrar a conformidade com os requisitos;
        \subitem - O software integrado é verificado utilizando os critérios definidos;
        \subitem - Os resultados dos testes são armazenados;
        \subitem - Uma estratégia de regressão é desenvolvida para retestar o software quando há mudanças.
\end{itemize}


A partir dos propósitos, resultados e atividades de cada processo da norma ISO/IEC 12207/2008, foram levantadas as ações de melhoria
para alcançar os objetivos de melhoria propostos, que estão descritas na próxima seção.

Além dos processos da norma ISO/IEC 12207/2008, alguns princípios e práticas ágeis do \textit{eXtreme Programming} (XP) \cite{xp} 
foram levados em conta para embasar a definição das ações de melhoria. São eles:

\begin{itemize}
 \item Rápido \textit{Feedback};
 \item Manter a simplicidade;
 \item Mudanças incrementais;
 \item Trabalho de qualidade;
 \item Testes;
 \item Refatoração;
 \item Integração contínua;
 \item Releases curtas.
\end{itemize}

\vfill
\pagebreak
\section{Ações de melhoria}

Com os objetivos de melhoria definidos na fase de Iniciação foram estabelecidas ações para alcançá-los. Estas ações estão 
descritas na Tabela \ref{tab:acoes}. Os objetivos de melhoria foram priorizados de acordo com as necessidades da organização
alinhadas com a experiência da equipe de desenvolvimento no contexto.

\begin{table}[!h]
\centering
\caption{Ações de melhoria}
\label{tab:acoes}
\begin{tabular}{|c|c|l|l|}
\hline
\multicolumn{1}{|l|}{Prioridade} & \multicolumn{1}{l|}{Objetivo}                                                                                    & Ações                                                                                       & Produto                                                 \\ \hline
\multirow{4}{*}{1}               & \multirow{4}{*}{\begin{tabular}[c]{@{}c@{}}Formalizar o processo de \\ desenvolvimento de software\end{tabular}} & \begin{tabular}[c]{@{}l@{}}Implantar boas práticas de\\  métodos ágeis\end{tabular}         & \multicolumn{1}{c|}{\multirow{4}{*}{Processo modelado}} \\ \cline{3-3}
                                 &                                                                                                                  & \begin{tabular}[c]{@{}l@{}}Padronizar a execução das \\ atividades do processo\end{tabular} & \multicolumn{1}{c|}{}                                   \\ \cline{3-3}
                                 &                                                                                                                  & Modelar o processo                                                                          & \multicolumn{1}{c|}{}                                   \\ \cline{3-3}
                                 &                                                                                                                  & Validar o processo                                                                          & \multicolumn{1}{c|}{}                                   \\ \hline
\multirow{3}{*}{2}               & \multirow{3}{*}{\begin{tabular}[c]{@{}c@{}}Otimizar a atividade de \\ implantação\end{tabular}}                  & \begin{tabular}[c]{@{}l@{}}Definir atividades rigorosas \\ de implantação\end{tabular}      &                                                         \\ \cline{3-4} 
                                 &                                                                                                                  & \begin{tabular}[c]{@{}l@{}}Selecionar ferramenta de\\  integração contínua\end{tabular}     &                                                         \\ \cline{3-4} 
                                 &                                                                                                                  & Implantar integração contínua                                                               & Script                                                  \\ \hline
\multirow{2}{*}{3}               & \multirow{2}{*}{\begin{tabular}[c]{@{}c@{}}Estabelecer métricas \\ para o processo\end{tabular}}                 & Definir medições relevantes                                                                 &                                                         \\ \cline{3-4} 
                                 &                                                                                                                  & Definir plano de medição                                                                    & Plano de Medição                                        \\ \hline
\end{tabular}
\end{table}

Para o objetivo de melhoria "\emph{Formalizar o processo de desenvolvimento de software}", no processo a ser modelado as
seguintes atividades devem ser incluídas, baseando-se nas atividades dos processos de Garantia de Qualidade de Software,
de Gestão de Configuração e de Teste de Qualificação de Software da norma ISO/IEC 12207/2008 \cite{12207}:
 
\begin{itemize}
  \item Definir um subprocesso de garantia de qualidade;
    \subitem - Definir atividades para Garantia do Produto;
    \subitem - Definir atividades para Garantia do Processo;
    \subitem - Adicionar atividades de Verificação e Validação para a garantia da qualidade.
  \item Definir e padronizar a estratégia de gerência de configuração;
    \subitem - Definir estratégia para o controle de mudanças;
  \item Estabelecer e manter suíte de testes;
    \subitem - Desenvolver critérios de aceitação para o software integrado, de acordo com os requisitos;
    \subitem - Montar suíte de testes;
    \subitem - Definir estratégia para testes de regressão;
    \subitem - Atualizar suíte de testes e documentação associada.
    \subitem - Documentar resultados dos testes;
\end{itemize}

O objetivo de melhoria "\emph{Otimizar a atividade de implantação}" consiste em facilitar a atividade de implantação,
para que o impacto da burocracia no acesso ao ambiente de produção seja mínimo no desenvolvimento do sistema.
Para atingir este objetivo, é proposta a adição da técnica de integração contínua ao processo para automatizar a
implantação e atualização do código. Esta proposta está alinhada com os princípios e práticas ágeis Rápido \textit{feedback},
Testes e Integração Contínua, propostos por \citeonline{xp} na metodologia XP.

O objetivo de melhoria "\emph{Estabelecer métricas para o processo}" seria basicamente um subprocesso 
do processo que será obtido com o objetivo "\emph{Formalizar o processo de desenvolvimento de software}", 
mas a sua importância para o contexto é tamanha que este foi destacado como um objetivo de melhoria por si só,
pois para que a melhoria possa ser visualizada é preciso métricas para acompanhar a evolução do processo.
Para este objetivo de melhoria, as seguintes atividades devem ser incluídas ao processo de medição,
baseado no processo de Gestão de Medição da norma ISO/IEC 12207/2008 \cite{12207}:

\begin{itemize}
 \item Identificar e priorizar as necessidades de informação;
 \item Selecionar e documentar as métricas a serem coletadas, que satisfaçam as necessidades de informação;
 \item Definir procedimentos de medição;
 \item Executar o processo de medição, seguindo os procedimentos;
    \subitem - Coletar, armazenar e verificar dados;
    \subitem - Analisar os dados e desenvolver produtos de informação;
    \subitem - Comunicar resultados;
 \item Avaliar produtos de informação, identificar e comunicar possíveis melhorias;
\end{itemize}

Para a definição das métricas o método GQM (Basili et al, 1994 apud \cite{solingen99}) se mostra adequado, 
pois apresenta um processo de medição bem estruturado, detalhado, dividido em diferentes níveis de abstração
e orientado a objetivos, também estando em conforme com as especificações do processo de Gestão de Medição
da norma ISO/IEC 12207/2008.

\section{Planejamento da avaliação}
    
    Para avaliar o resultado do processo de melhoria, serão coletadas as seguintes métricas:
    
    \begin{itemize}
      \item Número de desvios ao processo;
      \item Número de defeitos por atualização de código;
      \item Número de defeitos decorrentes da implantação;
      \item Cobertura de código;
    \end{itemize}
    
    A avaliação das melhorias será comparada com valores base das métricas acima. Para determinar os valores base para comparação, 
    as métricas definidas acima serão coletadas em um período de duas \textit{sprints} e será realizada uma média dos valores.
    
    Como o desenvolvimento do SiGA segue o \textit{framework Scrum}, as ações implementadas serão avaliadas semanalmente,
    no fim de cada \textit{Sprint}, a partir da coleta das métricas definidas acima e comparação com os valores base.