\chapter{Processo de Desenvolvimento Atual}
\label{ap:processo_atual}

Neste capítulo estão definidos os papéis, atividades e ferramentas utilizadas no processo de desenvolvimento atual do SiGA.

\section*{Papéis}

O processo é realizado pelos seguintes papéis:

\begin{itemize}
 \item \textbf{Gerente/Analista de Requisitos:} Responsável por levantar os requisitos com o cliente, gerenciar a equipe, gerenciar o custo do projeto, 
 gerenciar o cronograma do projeto e realizar a implantação do sistema;
 \item \textbf{Desenvolvedores:} Desenvolver e testar o sistema;
 \item \textbf{Time:} Analisar os requisitos;
\end{itemize}

\section*{Ferramentas}

O processo é apoiado pelas seguintes ferramentas:

\begin{itemize}
 \item \textbf{Sublime Text:} Editor de texto utilizado para escrita do código; 
 \item \textbf{phpMyAdmin:} Gerenciador de banco de dados;
 \item \textbf{Github}: Utilizado para versionamento de código e gerenciamento de requisitos;
 \item \textbf{Waffle}: Utilizado para gerenciamento de requisitos de acordo com Kanban \cite{kanban};
\end{itemize}

\section*{Atividades}

O processo é composto de dez atividades que estão detalhadas nas tabelas abaixo.

\begin{table}[]
\centering
\caption{Descrição da atividade - Levantar Requisitos}
\begin{tabular}{|l|l|}
\hline
\textbf{Atividade}     & Levantar Requisitos                                                                                                                                                                                    \\ \hline
\textbf{Objetivo}      & Conhecer as necessidades do cliente a serem contempladas pelo sistema                                                                                                                                  \\ \hline
\textbf{Entradas}      & Escopo do projeto                                                                                                                                                                                      \\ \hline
\textbf{Saídas}        & Requisitos                                                                                                                                                                                             \\ \hline
\textbf{Responsável}   & Gerente/Analista de Requisitos                                                                                                                                                                         \\ \hline
\textbf{Procedimentos} & \begin{tabular}[c]{@{}l@{}}Não se aplica.\\ \\ Observações: A execução desta atividade não demanda procedimentos \\ definidos pois os requisitos podem ser levantados de diversas formas.\end{tabular} \\ \hline
\end{tabular}
\end{table}

\begin{table}[]
\centering
\caption{Descrição da atividade - Validar Requisitos}
\begin{tabular}{|l|l|}
\hline
\textbf{Atividade}     & Validar Requisitos                                                                                                                                                                                    \\ \hline
\textbf{Objetivo}      & Validar os requisitos levantados com o cliente                                                                                                                                                        \\ \hline
\textbf{Entradas}      & Requisitos                                                                                                                                                                                            \\ \hline
\textbf{Saídas}        & Requisitos refinados                                                                                                                                                                                  \\ \hline
\textbf{Responsável}   & Gerente/Analista de Requisitos                                                                                                                                                                        \\ \hline
\textbf{Procedimentos} & \begin{tabular}[c]{@{}l@{}}Não se aplica.\\ \\ Observações: A execução desta atividade não demanda procedimentos \\ definidos pois os requisitos podem ser validados de diversas formas.\end{tabular} \\ \hline
\end{tabular}
\end{table}

\begin{table}[]
\centering
\caption{Descrição da atividade - Priorizar Requisitos}
\begin{tabular}{|l|l|}
\hline
\textbf{Atividade}     & Priorizar Requisitos                                     \\ \hline
\textbf{Objetivo}      & Priorizar os requisitos definidos para o desenvolvimento \\ \hline
\textbf{Entradas}      & Requisitos                                               \\ \hline
\textbf{Saídas}        & Requisitos priorizados (a serem desenvolvidos na Sprint) \\ \hline
\textbf{Responsável}   & Gerente/Analista de Requisitos                           \\ \hline
\textbf{Procedimentos} & Não se aplica.                                           \\ \hline
\end{tabular}
\end{table}

\begin{table}[]
\centering
\caption{Descrição da atividade - Especificar e Documentar Requisitos}
\begin{tabular}{|l|l|}
\hline
\textbf{Atividade}     & Especificar e Documentar Requisitos                                                                                                                                                                                       \\ \hline
\textbf{Objetivo}      & \begin{tabular}[c]{@{}l@{}}Especificar os requisitos levantados detalhando a funcionalidade\\ e os critérios de aceitação e estimar.\end{tabular}                                                                         \\ \hline
\textbf{Entradas}      & Requisitos                                                                                                                                                                                                                \\ \hline
\textbf{Saídas}        & Issues                                                                                                                                                                                                                    \\ \hline
\textbf{Responsável}   & Time                                                                                                                                                                                                                      \\ \hline
\textbf{Procedimentos} & \begin{tabular}[c]{@{}l@{}}1. Discutir os requisitos levantados\\ 2. Escrever a história de usuário\\ 3. Escrever os critérios de aceitação\\ 4. Definir o responsável\\ 4. Definir a pontuação para a issue\end{tabular} \\ \hline
\end{tabular}
\end{table}